\documentclass{beamer}

\usepackage{amsmath}
\usepackage{listings}
\usepackage{pstricks,pst-node,pst-text,pst-3d}
\usepackage{multicol}

\mode<presentation>
{
  \usetheme{Warsaw}
  %\useoutertheme{default}
%	\useinnertheme{circles}
%  \usecolortheme{default}
  %\setbeamercovered{transparent}
}

\definecolor{Brown}{cmyk}{0,0.81,1,0.60}
\definecolor{OliveGreen}{cmyk}{0.64,0,0.95,0.40}
\definecolor{CadetBlue}{cmyk}{0.62,0.57,0.23,0}
\lstset{ %
language=TeX,                % choose the language of the code
basicstyle=\footnotesize,       % the size of the fonts that are used for the code
backgroundcolor=\color{white},  % choose the background color. You must add \usepackage{color}
showspaces=false,               % show spaces adding particular underscores
showstringspaces=false,         % underline spaces within strings
stringstyle=\ttfamily,
showtabs=false,                 % show tabs within strings adding particular underscores
frame=none,                   % adds a frame around the code
tabsize=2,                      % sets default tabsize to 2 spaces
captionpos=b,                   % sets the caption-position to bottom
breaklines=true,                % sets automatic line breaking
breakatwhitespace=false,        % sets if automatic breaks should only happen at whitespace
commentstyle=\color{CadetBlue},
identifierstyle=\ttfamily\color{Brown}\bfseries,
keywordstyle=\ttfamily\color{OliveGreen},
escapeinside={\%*}{*)}          % if you want to add a comment within your code
}


\newenvironment{specialframe}{\begin{frame}[fragile,environment=specialframe]}{\end{frame}}

\begin{document}

\title[]
{{\AE}ther: Digital Interactivity System}

%\subtitle
%{An even briefer subtitle}

\date {\today}

\author[Colorado State University]
{Adam Labadorf and Alan LaMielle} % (optional)

\begin{frame}
    \titlepage
\end{frame}

\begin{frame}{What is {\AE}ther?}

\begin{itemize}
\item Framework for easily developing interactive applications (i.e. {\it games})
\item Utilizes input from diverse sources (e.g. camera, IR, accelerometer, \& c.)
\item Rapid prototyping environment for external input sources
\end{itemize}

\end{frame}



\begin{frame}{What does it do?}

\begin{itemize}

\item Use your shadow as input to a computer program 
\begin{itemize}
\item Fun (EncouragerModule, CatchOSIfUCan, TileMosaic)
\item Physics simulations
\end{itemize}

\item Laser pointer detection:
\begin{itemize}
\item Control presentations (you're watching it!)
\item Play space-invaders type games
\end{itemize}

\item Face Detection:
\begin{itemize}
\item Put masks on people (TikiMasks)
\item Show who's been in a building (FaceTiler)
\item Which faculty member do you most look like?
\end{itemize}

\end{itemize}

\end{frame}



\begin{frame}{How does it work?}

\begin{itemize}
\item Written in {\tt Python} using the {\tt pygame} package and {\tt opencv}
\item Developers write {\it modules} that fit into the {\AE}ther framework
\item Input is provided through {\it chains} composed of transforms, which hide the details of input capture
\end{itemize}

\end{frame}



\begin{frame}{Why is this Computer Science?}

\begin{itemize}
\item Intersection of many computer related fields including (but not limited to):
\begin{itemize}
\item Computer vision and graphics
\item Software engineering
\item Human Computer Interaction
\item Game development
\end{itemize}
\item Provides an easy way to develop applications with unconventional/novel input methods
\item Implements good software engineering and project management practices
\begin{itemize}
\item version control
\item bug tracking
\item modular code design
\end{itemize}
\item It's really cool

\end{itemize}

\end{frame}



\begin{frame}

\begin{center}
\huge
THIS.....is Computer Science
\end{center}

\end{frame}


\end{document}
